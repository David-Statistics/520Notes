\documentclass[]{book}
\usepackage{lmodern}
\usepackage{amssymb,amsmath}
\usepackage{ifxetex,ifluatex}
\usepackage{fixltx2e} % provides \textsubscript
\ifnum 0\ifxetex 1\fi\ifluatex 1\fi=0 % if pdftex
  \usepackage[T1]{fontenc}
  \usepackage[utf8]{inputenc}
\else % if luatex or xelatex
  \ifxetex
    \usepackage{mathspec}
  \else
    \usepackage{fontspec}
  \fi
  \defaultfontfeatures{Ligatures=TeX,Scale=MatchLowercase}
\fi
% use upquote if available, for straight quotes in verbatim environments
\IfFileExists{upquote.sty}{\usepackage{upquote}}{}
% use microtype if available
\IfFileExists{microtype.sty}{%
\usepackage{microtype}
\UseMicrotypeSet[protrusion]{basicmath} % disable protrusion for tt fonts
}{}
\usepackage[margin=1in]{geometry}
\usepackage{hyperref}
\hypersetup{unicode=true,
            pdftitle={STAT520 Fall 2016 Notes},
            pdfauthor={David Clancy},
            pdfborder={0 0 0},
            breaklinks=true}
\urlstyle{same}  % don't use monospace font for urls
\usepackage{natbib}
\bibliographystyle{apalike}
\usepackage{longtable,booktabs}
\usepackage{graphicx,grffile}
\makeatletter
\def\maxwidth{\ifdim\Gin@nat@width>\linewidth\linewidth\else\Gin@nat@width\fi}
\def\maxheight{\ifdim\Gin@nat@height>\textheight\textheight\else\Gin@nat@height\fi}
\makeatother
% Scale images if necessary, so that they will not overflow the page
% margins by default, and it is still possible to overwrite the defaults
% using explicit options in \includegraphics[width, height, ...]{}
\setkeys{Gin}{width=\maxwidth,height=\maxheight,keepaspectratio}
\IfFileExists{parskip.sty}{%
\usepackage{parskip}
}{% else
\setlength{\parindent}{0pt}
\setlength{\parskip}{6pt plus 2pt minus 1pt}
}
\setlength{\emergencystretch}{3em}  % prevent overfull lines
\providecommand{\tightlist}{%
  \setlength{\itemsep}{0pt}\setlength{\parskip}{0pt}}
\setcounter{secnumdepth}{5}
% Redefines (sub)paragraphs to behave more like sections
\ifx\paragraph\undefined\else
\let\oldparagraph\paragraph
\renewcommand{\paragraph}[1]{\oldparagraph{#1}\mbox{}}
\fi
\ifx\subparagraph\undefined\else
\let\oldsubparagraph\subparagraph
\renewcommand{\subparagraph}[1]{\oldsubparagraph{#1}\mbox{}}
\fi

%%% Use protect on footnotes to avoid problems with footnotes in titles
\let\rmarkdownfootnote\footnote%
\def\footnote{\protect\rmarkdownfootnote}

%%% Change title format to be more compact
\usepackage{titling}

% Create subtitle command for use in maketitle
\newcommand{\subtitle}[1]{
  \posttitle{
    \begin{center}\large#1\end{center}
    }
}

\setlength{\droptitle}{-2em}
  \title{STAT520 Fall 2016 Notes}
  \pretitle{\vspace{\droptitle}\centering\huge}
  \posttitle{\par}
  \author{David Clancy}
  \preauthor{\centering\large\emph}
  \postauthor{\par}
  \predate{\centering\large\emph}
  \postdate{\par}
  \date{August - December 2016}

\usepackage{booktabs}

\let\BeginKnitrBlock\begin \let\EndKnitrBlock\end
\begin{document}
\maketitle

{
\setcounter{tocdepth}{1}
\tableofcontents
}
\chapter{Intro}\label{intro}

\chapter{Chap 1}\label{ch1}

\chapter{Literature}\label{literature}

\chapter{Chapter 3 - Common Families of Distributions}\label{ch3}

\section{9/21/2016}\label{section}

\BeginKnitrBlock{rmddefinition}
A \emph{single distribution} is completely specified
(e.g.~Gamma(3,2)).\\
A \emph{family of distributions} is defined by a functional form and
parameter space containing more than 1 element
(e.g.~Uniform(\(a\),\(b\));
\(\Theta = \{(a,b) : a,b, \in \mathbb{R}\})\)
\EndKnitrBlock{rmddefinition}

\subsection{\texorpdfstring{Discrete Uniform
\((N)\)}{Discrete Uniform (N)}}\label{discrete-uniform-n}

\[P(X = x) = \frac{1}{N}; x \in \{1,2,3,...\}\]

Typical example: Die\\
\[
\begin{aligned}
  E[X] &= \sum_{i=1}^N i\frac{1}{N} \\
    &= \frac{1}{N}\frac{N(N+1)}{2} \\
    &=\frac{N+1}{2} \\
  E[X^2] &= \sum_{i=1}^N i^2\frac{1}{N} \\
    &= \frac{1}{N}\frac{N(N+1)(2N + 1)}{6} \\
    &= \frac{(N+1)(2N+1)}{6} \\
  V[X] &= E[X^2] - E[X]^2 \\
    &= \frac{(N+1)(2N+1)}{6} - \frac{(N+1)^2}{4} \\
    &= \frac{2(N+1)(2N+1) - 3(N+1)^2}{12} \\
    &= \frac{(N+1)((4N+2) - (3N+3))}{12} \\
    &= \frac{(N+1)(N-1)}{12}
\end{aligned}
\]

\subsection{\texorpdfstring{Hypergeometric
\((R,W,n)\)}{Hypergeometric (R,W,n)}}\label{hypergeometric-rwn}

Suppose an urn has \(R\) red balls and \(W\) white balls and suppose
\(n\) balls are sampled without replacement. Let \(T_n\) be the number
of red balls sampled. \[
\begin{aligned}
  T_n &= \mbox{Hypergeometric}(R,W,n) \\
  P(T_n = k) &= \frac{{R \choose k}{W \choose n-k}}{{R + W \choose n}}; k \in \{ \max(0, n-W), ..., \min(R,n)\}
\end{aligned}
\] Note that if we see \(T_n = X_1 + ... + X_n\), the \(X_i\) are
\textbf{not} independent. \[
\begin{aligned}
  E[T_n] &= E[\sum X_i] = \frac{nR}{R+W} \\
  V[T_n] &= np(1-p)\left(1 - \frac{n-1}{R+W-1}\right)
\end{aligned}
\] In the variance, \(np(1-p)\) is the binomial variance while
\(\left(1 - \frac{n-1}{R+W-1}\right)\) is the \emph{finite sample
correction}.

\BeginKnitrBlock{rmdexample}
Last Christmas, the kitchen had 10 white lights and the living room had
20 colored lights 5 of the 30 failed. What is the probability exactly 3
were colored?
\EndKnitrBlock{rmdexample}

Urn1:

\[P(T_5 = 3) = \frac{{20 \choose 3} {10 \choose 2}}{{30 \choose 5}} \]

This is a Hypergeometric(20,10,5) distribution and we're sampling the
failed lights.

Urn2:

\[ P(T_{20} = 3) = \frac{{5 \choose 3} {25 \choose 17}}{{30 \choose 20}}\]

This is a Hypergeometric(5,25,20) distribution and we're sampling the
white lights.

\chapter{Applications}\label{applications}

\chapter{Final Words}\label{final-words}

\chapter{Placeholder}\label{placeholder}

\bibliography{packages.bib,book.bib}


\end{document}
